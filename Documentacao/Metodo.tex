\section{Método}\label{Metodo}
O modelo dos sapos pode ser resumido nas equações abaixo, onde x corresponde ao número de vocalizantes e y ao número de vocalizantes facultativos.

Ainda é importante lembrar que a população é fixa, como havíamos discutido, logo, no código do R é natural entender a substituição do x por N-y, onde N é o número total da população, ou seja, N=x + y. Vale lembrar que por conta dessa premissa, não estão incluindo os termos de nascimento e morte explicitamente. Supostamente a mesma quantidade de animais que morrem, nascem, sem alterar a proporção de vocalizantes e não-vocalizantes.

\subsection{Modelo Baseado em Exponencial}
O modelo baseado na exponencial é mais complexo no sentido de determinar relações analíticas. Vou escrever na forma de equações diferenciais ordinárias acopladas, por conta da facilidade de comunicação.

\vspace{3 mm}
(1) $\frac{dx}{dt} = k_1 y(t) -k_2 x(t)$

\vspace{3 mm}
(2) $\frac{dy}{dt}= e^{-\rho y} + (1-k_3 e^{-\lambda x}) -k_1 y(t) +k_2 x(t)$
\vspace{3 mm}

Onde o termo $e^{-\rho y}$ é o sucesso reprodutivo de x, o termo $(1-k_3 e^{-\lambda x})$ é o sucesso reprodutivo de y e o $k_3$ está relacionado com o sucesso reprodutivo de y na ausência de $x$.

Ainda, o modelo pode ser escrito na forma com a qual o Gillespie (criador da técnina) escreve:

\vspace{3 mm}
Reaction channel
\vspace{3 mm}

Para Vocalizante e não-vocalizantes (1)

R1: Não-vocalizante que se torna vocalizante

R2: Vocalizante que se torna não-vocalizante

\vspace{3 mm}
Exclusivo para não-vocalizante (2)

R3: Sucesso reprodutivo de x, com parâmetro $\rho$

R4: Sucesso reprodutivo de y, com parâmetro $\lambda$

\vspace{3 mm}
Propensity functions

\vspace{3 mm}
Para vocalizante e não-vocalizantes (1)

\vspace{3 mm}
$a_1(y)=k_1 y(t)$

\vspace{3 mm}
$a_2(x)=k_2 x(t)$

\vspace{3 mm}
Para não-vocalizante (2)

\vspace{3 mm}
$a_3(x,y)=e^{-\rho y}$

\vspace{3 mm}
$a_4(x,y)=(1-k_3 e^{-\lambda x})$

\subsection{Modelo Sem Exponencial}
Este modelo é formalizado em termos de equações diferenciais ordinárias acopladas, por conta da facilidade de comunicação. Esta primeira forma do modelo foi desenvolvida do pelo Zé e lembra o sistema Lotka-Voterra (presa e predador).

Aqui serão descritos os pontos de equilíbrio e relações entre os parâmetros

\vspace{3 mm}
$\frac{dx}{dt} = k_1 y - k_2 x$

\vspace{3 mm}
$\frac{dy}{dt} = k_3 x - k_4 y x + k_5 y + k_6 x y + k_2 x - k_1 y$

\vspace{3 mm}
$\frac{dy}{dt} = (k_3 - k_4 y)x + (k_5 + k_6 y)x + \ldots$, é igual ao seu modelo, zé. Depois a gente arranca essa linha.

\vspace{3 mm}
Como: $k_c= k_4+k_6$

\vspace{3 mm}
$\frac{dy}{dt} = k_3 x - k_c y x + k_5 y + k_2 x - k_1 y$

Com essas condições temos os seguintes pontos de equilíbrio:

$y^* = \frac{k_1 k_3 + k_2 k_5}{k_c k_1}$

\vspace{3 mm}
$x^* = \frac{k_1 k_3 + k_2 k_5}{k_c k_2}$

O que mostra que efeito do $k_c$ realmente precisa ser negativo.

Esse P.E. é estável se:


$k_1 + k_2 > \frac{k_5(k_1 k_3 + k_2 k_5)}{k_2}$
\vspace{3 mm}

Rearranjando:

\vspace{3 mm}
$k_2(k_1 + k_2) > k_5(k_1 k_3 + k_2 k_5)$

Finalmente, esse P.E. se comporta como um nó se:

$k^2_1 + k_2^2 + \frac{k_5^2 k_1^2 k_3^2}{k_2^2} +2 \frac{k_5^3 k_1 k_3}{k_2} +k_5^4 > $
$ 6 k_1 k_2 + 2 \frac{k_5 k_1^2 k_3}{k_2} + 2 k_1 k_5^2 + 2k_1 k_3 k_5 + 2 k_2 k_5^2 + 4 k_1 k_3 + 4 k_2 k_5$ 

Caso contrário, é um foco.

É importante lembrar que $x^*=0$ e $y^*=0$ também são P.E. e são estáveis nessas condições:

\vspace{3 mm}
$k_1 + k_2 > k_5$
$k_1 k_3 + 2k_1 k_2 > k_2 k_5$

Porém, para o nosso caso, é interessante que eles sejam instáveis.

\subsubsection{Considerações sobre o modelo sem exponencial}
Nessa formulação existe um efeito que eu não sei se é interessante para o modelo de estratégia dos sapos.

O termo $-k_4 yx$ corresponde ao efeito de y no sucesso reprodutivo de $x$, faz sentido? Bom, se isso fizer sentido, ao estudar o efeito deste termo com a variação de $y$, com $y \in [0,N]$, o máximo acontece quando $y=x=\frac{N}{2}$, não esquecendo que N é constante.

O resultado disso é que $y > \frac{N}{2}$ o efeito $-k_4 yx$ volta a ficar pequeno, pois o máximo desse efeito aconteceu quando $y=\frac{N}{2}$, certo?