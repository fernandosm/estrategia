\section{Método}\label{Metodo}

Depois da reunião com o Zé, notei que os modelos abaixo estão com algumas coisas erradas que precisam ser alterados!!! Repensar nas equações do sucesso reprodutivo do x na presença do y.

O modelo dos sapos pode ser resumido nas equações abaixo, onde x corresponde ao número de vocalizantes e y ao número de vocalizantes facultativos (aqui, é para entender vocalizante parcial). Notem que tem  equações diferentes para y, pois ainda não sabemos qual usar (só testei a a).

\begin{center}
\vspace{3 mm}
$x(t + \tau)= +x(t) +k_1 y(t) -k_2 x(t)$ (1)

\vspace{3 mm}
$y(t + \tau)= +y(t) + (1-e^{-\lambda y(t)})x(t) + (1-e^{-\gamma x(t)})y(t) -k_1 y(t) +k_2 x(t)$ (2a)

\vspace{3 mm}
$y(t + \tau)= +y(t) + (1-e^{-\lambda x(t)y(t)}) + (1-e^{-\gamma x(t)y(t)}) -k_1 y(t) +k_2 x(t)$ (2c)
\end{center}

Ainda, o modelo pode ser escrito na forma com a qual o Gillespie (criador da técnina) escreve

Reaction channel

Para Vocalizante e não-vocalizantes (1)

R1: Não-vocalizante que se torna vocalizante

R2: Vocalizante que se torna não-vocalizante

Exclusivo para não-vocalizante (2a), (2b) e (2c)

R3: Sucesso reprodutivo de x, com parâmetro lambda

R4: Sucesso reprodutivo de y, com parâmetro gama

Propensity functions

Para vocalizante e não-vocalizantes (1)

\begin{center}
\vspace{3 mm}
$a_1(y)=k_1 y(t)$

\vspace{3 mm}
$a_2(x)=k_2 x(t)$
\end{center}

Para não-vocalizante (2a)

\begin{center}
\vspace{3 mm}
$a_3(x,y)=(1-e^{-\lambda y(t)})x(t)$

\vspace{3 mm}
$a_4(x,y)=(1-e^{-\gamma x(t)})y(t)$
\end{center}

ou para não-vocalizante (2b)

\begin{center}
\vspace{3 mm}
$a_3(x,y)=1-e^{-\lambda x(t)y(t)}$

\vspace{3 mm}
$a_4(x,y)=1-e^{-\gamma x(t)y(t)}$
\end{center}

Ainda é importante lembrar que a população é fixa, como havíamos discutido, logo, no código do R é natural entender a substituição do x(t) por N-y(t), onde N é o número total da população, ou seja, N=x(t) + y(t). Vale lembrar que por conta dessa premissa, não estão incluindo os termos de nascimento e morte explicitamente. Supostamente a mesma quantidade de animais que morrem, nascem, sem alterar a proporção de vocalizantes e não-vocalizantes.